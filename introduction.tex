\chapter*{Introduction}		 		 		 % DO NOT TOUCH!
\addcontentsline{toc}{chapter}{Introduction} % DO NOT TOUCH!

The procedure of finding answers to questions that arise in scientific and engineering disciplines often begins by defining what the problem is. The next step is to formulate and present the problem in such a way that allows for a simple and accurate solution. One of the most common ways to represent a problem is by using a system of questions, for example, linear or partial differential equations. The former systems can be found in various fields such as engineering, physics, computer science, and economics. The latter systems regularly emerge in both physics and engineering, for example, electrodynamics, fluid dynamics, thermodynamics, etc. Apart from emerging in similar fields, these two systems of equations can both be represented by matrices and thus, solved using basic tools of linear algebra.
\par Consequently, from the perspective of computer science, such problems can be solved using many different methods, for example, direct methods. This group of methods is characterized by finding the solution to a problem by a finite sequence of operations. While such methods can - theoretically - provide an exact solution, in reality, due to rounding errors, their accuracy may not always be adequate. Furthermore, owing to the methods' sequential nature, they can seldom be parallelized and as such they are often run on a Central Processing Unit (CPU) which can take a considerable amount of time. For these reasons, an alternative means of arriving at a solution can be found in iterative methods which use an initial value as a starting point and then iteratively converge to an approximate solution. One of the advantages of using iterative methods over direct methods is that their procedures can usually be parallelized. This, combined with the rise in popularity of utilizing Graphics Processing Units (GPUs) for scientific computations, led to the tailoring of parallelizable algorithms for optimal execution times on the GPU. This project aims to repurpose a direct method which can be used to solve systems of equations (LU decomposition) into an iterative method, and, furthermore, to parallelize its execution on the GPU.
\par The first chapter presents a detailed description of the hardware of GPUs along with the software that is used to leverage their computational power. Additionally, the LU decomposition method is introduced. The second chapter describes the implementation of the project that houses the direct and iterative versions of the LU decomposition method and any accompanying functionalities. Finally, the last chapter presents the results of benchmarks that were run in order to compare the CPU and GPU implementations.